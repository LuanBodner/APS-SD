\documentclass[12pt,a4paper]{article}
\usepackage[utf8]{inputenc}
\usepackage{amsmath}
\usepackage{amsfonts}
\usepackage{amssymb}
\begin{document}
A ferramenta Redis

Redis é comumente referido como um servidor de estrutura de dados. Isso significa que a feramenta da acesso à uma série de estruturas de dados mutáveis a partir de uma lista de comandos, enviadas por meio de um modelo cliente-servidor utilizando sockets TCP e um protocolo simples. Desta forma, diferentes processos podem enviar e modificar dados de forma compartilhada.

As estruturas implementadas no Redis possuem algumas propriedas especiais:

\begin{itemize}
\item Armazenamento em disco, mesmo que elas estejam modificadas e armazenadas no servidor. Desta forma, REDIS é capaz de manter sua eficiencia.
\item Redis oferece opções comuns em banco de dados, como replicação, níveis de durabilidade mutáveis , etc.
\end{itemize}

Outra forma de se analizar o Redis é enchergá-lo como uma versão mais complexa de memchaced, onde as operações são apenas SET e GET. Redis no caso, possui dados mais complexos e mais operações possíveis também.

Construindo Redis

Redis é uma ferramenta multiplataforma, mas o foco é mais forte no Linux. Redis dá suporrte a big endian e little endian e também a 32 e 64bits.

Para instalar a ferramenta, basta fazer o download no site e compilar, já que a ferramenta é distribuída como código fonte.

Assim, deve-se rodar o servidor e o cliente para fazer os teste na máquina.

Pub/Sub Redis

As funções SUBSCRIBE, UNSUBSCRIBE e PUBLISH são responsáveis pela implementação do paradigma Publish/Subscrive onde os senders não estão programados para mandar mensagens para receptores, mas para canais sem saber quais serão, ou se eles existem. Os inscritos expressam o interessa em um ou mais canais e apenas recebem mensgens de interesse. Esse paradigma permite maior scalabilidade e melhor topologia

Por Exemplo, para se inscrever a um canal foo e bar, o cliente manda um comando SUBSCRIBE  dando o nome para os canais:

SUBSCRIBE foo bar

Mensagens eviadas por outros clientes para esses canais serão enviados pelo Redis para todos os clientes inscritos.
Um cliente inscrito por um ou mais canais não devem enviar comandos, apesar de poderem se inscrever e cancelar a inscrição em outros canais. As respostas para estes comandos são enviados em forma de mensagens, para que o cliente possa ler as mensagens de forma coerente, o primeiro elemento da mensagem indica o tipo.


Formato das mensagens enviadas

\begin{itemize}
\item subscribe : significa que a inscrição ocorreu com sucesso dado pelo segundo elemento da mensagem. O terceiro elemento representa o numero de canais em que você está inscrito.
\item unsubscribe : significa que o cancelamento da inscrição ocorreu com sucesso dado pelo segundo elemento da mensagem. O terceiro elemento representa o numero de canais em que você está inscrito.
\item message : é a mensagem recebida como resultado de um comando PUBLISH enviado por outro cliente. O segundo elemento é o nome do canal enviado, e o terceiro é a mensagem.  
\end{itemize}



\end{document}